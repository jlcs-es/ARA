\documentclass{article}
\usepackage[spanish]{babel}
\usepackage[onehalfspacing]{setspace}
\usepackage[utf8]{inputenc}
\usepackage{amsmath}
\usepackage{amssymb}
\usepackage{verbatim}
\usepackage{graphicx}
\usepackage{listings}
\usepackage{fullpage}
\usepackage{color}
\usepackage{fancyvrb}
\usepackage{hyperref}

\definecolor{mygreen}{rgb}{0,0.6,0}
\definecolor{mygray}{rgb}{0.5,0.5,0.5}
\definecolor{mymauve}{rgb}{0.58,0,0.82}

\lstset{ %
	backgroundcolor=\color{white},   % choose the background color; you must add \usepackage{color} or \usepackage{xcolor}
	basicstyle=\footnotesize,        % the size of the fonts that are used for the code
	breakatwhitespace=false,         % sets if automatic breaks should only happen at whitespace
	breaklines=true,                 % sets automatic line breaking
	captionpos=b,                    % sets the caption-position to bottom
	commentstyle=\color{mygreen},    % comment style
	frame=single,                    % adds a frame around the code
	keepspaces=true,                 % keeps spaces in text, useful for keeping indentation of code (possibly needs columns=flexible)
	numbers=left,                    % where to put the line-numbers; possible values are (none, left, right)
	numbersep=5pt,                   % how far the line-numbers are from the code
	numberstyle=\tiny\color{mygray}, % the style that is used for the line-numbers
	rulecolor=\color{black},         % if not set, the frame-color may be changed on line-breaks within not-black text (e.g. comments (green here))
	showspaces=false,                % show spaces everywhere adding particular underscores; it overrides 'showstringspaces'
	showstringspaces=false,          % underline spaces within strings only
	showtabs=false,                  % show tabs within strings adding particular underscores
	stepnumber=1,                    % the step between two line-numbers. If it's 1, each line will be numbered
	stringstyle=\color{mymauve},     % string literal style
	tabsize=4,                       
	title=\lstname                   % show the filename of files included with \lstinputlisting; also try caption instead of title
}


\author{José Luis Cánovas Sánchez\\joseluis.canovas2@um.es\\48636907A}
\title{ARQUITECTURAS DE REDES AVANZADAS\\PRÁCTICA 1\\ ENERO 2015}
\date{}
\begin{document}
\maketitle

\begin{abstract}
	En este informe se redacta el desarrollo usando la herramienta GNS3 para el despliegue del escenario de red IPv6 con movilidad.
	%TODO: terminar
\end{abstract}

\tableofcontents
\section{Topología inicial}
Partimos en este proyecto creando una nueva topología en GNS3 equivalente a la del enunciado y que se puede ver en la imagen inferior. Para los routers se utiliza la ios del modelo C7200 de CISCO, con soporte para IPv6. Para los hosts, una imagen linux sin interfaz gráfica y con gestor de paquetes para poder instalar el soporte de mobile IPv6.\par

\begin{center} 
	\includegraphics[scale=0.5]{images/topologyInic.png}
\end{center}

La configuración básica para que todos los dispositivos y redes tengan conectividad IPv6 entre ellas se puede realizar por medio del fichero de configuración del router directamente, y su equivalente en comandos de la terminal suelen ser las mismas líneas aplicadas en el nivel \textit{configure terminal}. En cuanto a los hosts, como utilizan la autoconfiguración de IPv6 no hace falta configurar manualmente nada.
\\

Veamos primero la configuración común de todos los routers:

Para activar IPv6 es necesario indicarlo antes de la configuración de las interfaces con \textit{ipv6 unicast-routing}. En cada interfaz que activemos, debemos indicar \textit{ipv6 enable} y la orden \textit{ipv6 address diripv6}, donde \textit{diripv6} es la dirección IPv6 asignada manualmente, o el prefijo de red seguido de \textit{eui-64} para permitir que la interfaz se autoconfigure su dirección.

Con esto tendríamos activado IPv6 en cada router, pero al intentar hacer \textit{ping} a una interfaz de otra red, vemos que no se recibe respuesta. Esto es porque no hay información en las tablas de rutas más que de las redes conectadas directamente a las interfaces de cada router.

Para cada Sistema Autónomo vamos a usar OSPF con soporte de IPv6. El diseño será de una única área 0 para todo el AS, lo que incluye las subredes en AS\#13 3FE0:0:1::/64 y 3Fe0::/64, y en AS\#50 las subredes 2CAE:0:1::/64 y 2CAE::64. Para ello, debemos activar OSPF en las dos interfaces de los routers AR1 y AR2, y en la interfaz FastEthernet1/0 de los routers R1 y R2 que se pueden ver en la figura de la topología más arriba.                                                                         

Las órdenes para activar OSPF en cada router son:

Dentro de una interfaz que deba soportar OSPF (está dentro del área 0):
\begin{lstlisting}
ipv6 ospf 100 area 0
\end{lstlisting}

Tras la configuración de las interfaces:
\begin{lstlisting}
ipv6 router ospf 100
	router-id a.b.c.d
\end{lstlisting}

Donde \textit{a.b.c.d} es un valor identificador del router con 32 bits que se puede escribir como una dirección IPv4. Para cada router hemos elegido los identificadores \textit{1.1.1.1}, \textit{2.2.2.2}, \textit{3.3.3.3} y \textit{4.4.4.4} para R1, R2, AR1 y AR2 respectivamente. El valor 100 es indicador del proceso OSPF en el router y puede tomar cualquier otro valor, el tomar 100 es arbitrario.

Llegados a este punto, tenemos conectividad dentro de cada área, de modo que ahora un ping desde R1 a la interfaz de AR1 en la subred 3FE0::/64 funciona, pero de AR1 a R2 o AR2 no, porque no se conoce la ruta. Ahora mismo sólo R1 y R2 podrían hacer ping a las interfaces de la red 2001:ABCD::/64.

Para solucionarlo la forma más rápida es añadir una ruta estática en R1 y R2 a las redes del otro AS con la línea: \textit{ipv6 route 2CAE::/42 Serial2/0} en R1, y la equivalente \textit{ipv6 route 3FE0::/42 Serial2/0} en R2. Finalmente, en la configuración de OSPF, tras el \textit{router-id} añadimos la línea \textit{redistribute static}, sólo en R1 y R2, para que comuniquen con LSA5 (rutas fuera de dominio) que para alcanzar el otro AS a través de ellos hay una ruta.
\\

Ahora sí se puede realizar un ping desde cualquier dispositivo a cualquier otro, una vez OSPF ha convergido y compartido todos los mensajes. Pero como la mejora para configurar BGP entre R1 y R2 sobrescribe la configuración estática, mostraré las pruebas en el apartado siguiente.


\section{Configuración opcional BGP}
En los routers R1 y R2 debemos comentar o borrar las líneas de las rutas estáticas y su distribución en OSPF. A continuación, tras la configuración de OSPF podemos escribir la de BGP:
\\

Para R1:
\begin{lstlisting}
router bgp 13
	bgp router-id 1.1.1.1
	no bgp default ipv4-unicast
	!--- Without configuring ""no bgp default ipv4-unicast"" only IPv4 will be !--- advertised
	bgp log-neighbor-changes
	neighbor 2001:ABCD::2 remote-as 50
!
address-family ipv6
	neighbor 2001:ABCD::2 activate
	network 3FE0:0:1::/64
	network 3FE0::/64
exit-address-family
\end{lstlisting}

Donde 13 indica el número de Sistema Autónomo y 1.1.1.1 es un identificador equivalente al de OSPF, en este caso elegimos el mismo. La línea \textit{no bgp default ipv4-unicast} es necesaria para permitir las rutas con IPv6. La primera línea de \textit{neighbor} indica que en la interfaz de dirección 2001:ABCD::2, que corresponde a la IPv6 fija de R2, se encuentra el AS 50. A continuación, en el bloque \textit{address-family} activamos dicho vecino e indicamos que se deben anunciar las redes 3FE0:0:1::/64 y 3FE0::/64, correspondientes a las del AS 13 de la topología dada.

Para R2 la configuración sería equivalente cambiando los valores del 13 a 50 para el AS, el del \textit{router-id}, la dirección de R1 y las subredes a compartir.
\\

Con esto R1 y R2 conocerían las redes de cada AS, pero AR1 y AR2 todavía no, pues no hablan BGP. Sin embargo, sí hablan OSPF y añadiendo la línea en R1 y R2 \textit{redistribute bgp 10} en la configuración de OSPF (donde antes \textit{redistribute static}) se comunicarán las rutas externas con métrica 10 a los routers de cada AS. Podemos observarlo en la tabla de rutas de AR2, por ejemplo:

\begin{lstlisting}
Router>show ipv6 route
IPv6 Routing Table - 8 entries
Codes: C - Connected, L - Local, S - Static, R - RIP, B - BGP
U - Per-user Static route
I1 - ISIS L1, I2 - ISIS L2, IA - ISIS interarea, IS - ISIS summary
O - OSPF intra, OI - OSPF inter, OE1 - OSPF ext 1, OE2 - OSPF ext 2
ON1 - OSPF NSSA ext 1, ON2 - OSPF NSSA ext 2
C   2CAE::/64 [0/0]
via ::, FastEthernet2/0
L   2CAE::C801:1AFF:FE5F:38/128 [0/0]
via ::, FastEthernet2/0
C   2CAE:0:1::/64 [0/0]
via ::, FastEthernet1/0
L   2CAE:0:1:0:C801:1AFF:FE5F:1C/128 [0/0]
via ::, FastEthernet1/0
OE2  3FE0::/64 [110/2]
via FE80::C803:1AFF:FE5F:1C, FastEthernet1/0
OE2  3FE0:0:1::/64 [110/1]
via FE80::C803:1AFF:FE5F:1C, FastEthernet1/0
L   FE80::/10 [0/0]
via ::, Null0
L   FF00::/8 [0/0]
via ::, Null0
\end{lstlisting}                                                                                

En las líneas 16 a 19 se puede observar rutas a las redes del AS 13 conocidas por OSPF de rutas externas.
\\

Ahora las pruebas con ping y traceroute:

Ping y traceroute de AR1 a AR2 en la interfaz de la subred 2CAE::/64:
\begin{lstlisting}
Router>ping 2CAE::C801:1AFF:FE5F:38

Type escape sequence to abort.
Sending 5, 100-byte ICMP Echos to 2CAE::C801:1AFF:FE5F:38, timeout is 2 seconds:
!!!!!
Success rate is 100 percent (5/5), round-trip min/avg/max = 60/65/84 ms
\end{lstlisting}
\begin{lstlisting}
Router>traceroute 2CAE::C801:1AFF:FE5F:38

Type escape sequence to abort.
Tracing the route to 2CAE::C801:1AFF:FE5F:38

1 3FE0:0:1:0:C802:1AFF:FE5F:1C 16 msec 20 msec 20 msec
2 2001:ABCD::2 40 msec 40 msec 40 msec
3 2CAE::C801:1AFF:FE5F:38 60 msec 60 msec 60 msec
Router>
\end{lstlisting}
                                                       
Donde 3FE0:0:1:0:C802:1AFF:FE5F:1C es la IP autoconfigurada de R1 en la subred 3FE0:0:1::/64, 2001:ABCD::2 es la IP manual de R2 en la subred 2001:ABCD::/64 y finalmente 2CAE::C801:1AFF:FE5F:38 la IP de AR2.

Considerando que para llegar de AR1 a AR2 con ping y traceroute todos los routers deben conocer el camino en un sentido u otro, con esta única prueba podemos concluir que la configuración de la topología básica con OSPF y BGP es correcta.


\section{NEXT}



\begin{thebibliography}{99}
	\bibitem{apuntes}
	Apuntes
\end{thebibliography}


\end{document}
